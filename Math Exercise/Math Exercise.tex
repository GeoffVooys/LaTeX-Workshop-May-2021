\documentclass[10pt]{article}
\usepackage[alphabetic]{amsrefs}
\usepackage{amssymb}
\usepackage{amsfonts}
\usepackage{amsthm}
\usepackage{amsmath}
\usepackage{mathtools}
\usepackage{xy}


\numberwithin{Theorem}{section}


\title{Math Exercises}
\author{Geoff Vooys}
\date{\today}

\usepackage{graphicx}

\begin{document}

\maketitle

In the document below I've written some math down, but it looks ugly. Try to fix it up using what you've learned today!

The ring $A[\frac{x}{z}, \frac{y}{z}]$ for a commuting ring $A$ is helpful for thinking about projective space and the scheme $\mathbb{P}_A^2$.

Integrals are fun as long as we don't have to compute them. I like that for a group $G$ with left Haar measure $\mu$ we can say things like
\[
(\int_G f(gx)\,\mathrm{d}\mu(x))^{1/p} = (\int_G f(x)\,\mathrm{d}\mu(x))^{1/p}
\]
whenever $p \in (0,\infty)$ but man does it suck to compute these! However, we can use it to say neat things about the Banach spaces $L^p(G,\mu)$ and H{\"o}lder duality. When $p = 1$ we even get that $L^1(G,\mu)$ is a Banach algebra with multiplication given by the convolution $(f \ast g)(x) = \int_G f(y^{-1}x)g(y)\,\mathrm{d}\mu(y)$.

\end{document}